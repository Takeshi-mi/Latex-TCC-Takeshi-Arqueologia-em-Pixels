
O conteúdo abaixo é fruto de adaptações de um questionário feito no Google Forms e entrevistas orais realizadas pelo estudante Erik Takeshi Miura para o professor Edson Borges.


\begin{enumerate}
    \item \textbf{Qual é o objetivo principal do site reformulado?}  
    Divulgação dos resultados obtidos com as pesquisas já realizadas (e as futuras) sobre os sítios arqueológicos de Formosa.  
    Permitir que as pessoas acessem informações, relatórios, resultados, imagens, modelos 3D, ambiente virtual, relatórios e publicações.

    \item \textbf{Quem é o público-alvo do site?}  
    Pesquisadores/as, interessados/as em arqueologia, estudantes, turistas e demais pessoas interessadas.

    \item \textbf{Qual mensagem ou impressão o site deve transmitir?}  
    Transmitir a necessidade de estudo e preservação do patrimônio arqueológico da cidade de Formosa.  
    O site deve passar uma impressão de seriedade e sobriedade.

    \item \textbf{O conteúdo atual será mantido, atualizado ou substituído?}  
    Sim, o conteúdo atual será mantido, com atualizações e edições.

    \item \textbf{Quais tipos de conteúdo devem receber destaque no novo site?}  
    Imagens, modelos interativos e produção bibliográfica.

    \item \textbf{Há necessidade de novas seções ou categorias no site? Se sim, quais?}  
    Sim, mas ainda não foram definidas quais seções ou categorias serão criadas.

    \item \textbf{Quais problemas de design no site atual mais incomodam?}  
    Navegação, acessibilidade, inclusão, layout e diagramação são problemas atuais.  
    \textbf{Feedback recebido:} Usuários relataram dificuldade em encontrar informações devido ao site estar muito carregado e com seções bagunçadas.

    \item \textbf{Já foi recebido feedback dos usuários sobre o site atual? Se sim, qual?}  
    Sim. O feedback recebido foi que os usuários não conseguiram encontrar o que procuravam devido ao site estar muito carregado e com seções desorganizadas.

    \item \textbf{Existem sites ou referências visuais que servem como inspiração? Se sim, quais?}  
    Sim, os seguintes sites foram citados como referência:  
    \begin{itemize}
        \item \url{https://arqueologia-iab.com.br/}  
        \item \url{https://www.espacoarqueologia.com.br/}  
        \item \url{https://arqueologiaeprehistoria.com/o-que-e-arqueologia/}  
        \item \url{https://www.crypte.paris.fr/}  
        \item \url{https://www.louvre.fr/en/online-tours}  
        \item \url{https://www.japanhousesp.com.br/exposicao/nihoncha-introducao-ao-cha-japones/}
    \end{itemize}

    \item \textbf{Há uma paleta de cores ou identidade visual que deve ser seguida?}  
    Sim. A paleta de cores inclui tons ocres, vermelhos, pretos, cinzas e brancos.

    \item \textbf{Quais funcionalidades o site deve ter?}  
    \begin{itemize}
        \item Barra de busca  
        \item Botão de compartilhamento em redes sociais  
        \item Galeria de imagens  
        \item Integração com vídeos do YouTube ou Vimeo  
        \item Monitoramento de tráfego (ex.: Google Analytics)  
        \item Formulário para contato  
        \item Download de conteúdo (Modelos 3D, imagens, livros, simulações, etc.)  
        \item Ambiente virtual (não é uma plataforma de jogo, mas um ambiente interativo)  
        \item Outras funcionalidades que a imaginação permitir
    \end{itemize}

    \item \textbf{Será necessária autenticação de usuários?}  
    Sim, será necessária. Só pessoas com acesso podem modificar o site. Tem que ter uma área de administrador pra gerenciar o conteúdo.

    \item \textbf{Como podemos facilitar o acesso às informações para os usuários?}  
    Ampliação da divulgação e uso de palavras-chave adequadas para facilitar o acesso.  
    Melhorar a organização do conteúdo para evitar sobrecarga e confusão nas seções.

    \item \textbf{O site precisa ser acessível para pessoas com deficiência?}  
    Sim, o site precisa ser acessível para pessoas com deficiência.

    \item \textbf{Quais dispositivos e navegadores o público-alvo mais utiliza?}  
    Computadores desktop e smartphones.

    \item \textbf{Quem será responsável por atualizar o site no futuro?}  
    O cliente será responsável por atualizar o site no futuro.

    \item \textbf{Há necessidade de um sistema de gerenciamento de conteúdo (CMS)?}  
    Sim, há necessidade de um CMS.

    \item \textbf{Qual é a frequência esperada de atualizações no site?}  
    Provavelmente uma vez por ano, que é o tempo de um novo relatório de PIBIC ser publicado.

    \item \textbf{Existe alguma tecnologia ou ferramenta preferida para o desenvolvimento do site?}  
    Plataformas que possibilitem o uso de modelos 3D e ambiente virtual, como Unreal Engine ou outras.

    \item \textbf{Há necessidade de integração com outras plataformas? Se sim, quais?}  
    Não há necessidade de integração com outras plataformas.

    \item \textbf{Qual é o orçamento disponível para a manutenção e hospedagem do site a longo prazo?}  
    Não há orçamento definido, mas será avaliado o que é possível fazer. O orçamento é bastante restrito.

    \item \textbf{Como saberemos que o site reformulado está cumprindo seu objetivo?}  
    Creio que podemos deixar um pedido de avaliação da visitação, ou acompanhar os acessos.

    \item \textbf{Quais métricas ou indicadores de desempenho devem ser acompanhados?}  
    Contador de visitas (atualmente, o site tem cerca de 900 visualizações por ano, com 70 visualizações por mês).  
    Aumento de acessos pode ser esperado com divulgação em redes de ensino e turismo.
\end{enumerate}

