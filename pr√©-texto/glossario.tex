\chapter*{\centering Glossário}\label{glossario}
\addcontentsline{toc}{chapter}{Glossário}
\textbf{Acessibilidade}: Capacidade de um sistema, produto ou ambiente ser acessível e utilizável por pessoas com diferentes tipos de deficiência, garantindo inclusão digital.

\textbf{Ambiente Virtual}: Espaço digital criado para simular uma experiência imersiva e interativa, geralmente utilizado para exploração de modelos 3D ou cenários específicos.

\textbf{API}: Conjunto de protocolos e ferramentas que permitem a comunicação entre diferentes sistemas de software, facilitando a integração de funcionalidades externas em uma aplicação.

\textbf{Arqueologia Virtual}: Campo de estudo que utiliza tecnologias digitais para documentar, preservar e divulgar patrimônios arqueológicos de forma virtual.

\textbf{Avatar}: Representação digital de um personagem ou usuário em um ambiente virtual, frequentemente utilizado para interação ou navegação.

\textbf{Back-end}: Back-end é a parte interna de uma aplicação, que não é vista pelo usuário, mas que faz com que a aplicação funcione. É uma área de programação que envolve a lógica de processamento de dados, o banco de dados, as APIs e a infraestrutura de códigos. 

\textbf{Blueprints}: Sistema de programação visual da Unreal Engine que permite criar lógicas e funcionalidades utilizando nós.

\textbf{CMS Headless}: Sistema de gerenciamento de conteúdo que separa a camada de back-end (dados) da camada de front-end (interface), permitindo maior flexibilidade na entrega de conteúdo.

\textbf{Deploy}: Processo de publicar ou atualizar uma aplicação em um ambiente de produção, tornando-a acessível aos usuários finais.

\textbf{Design Responsivo}: Abordagem de design que garante que um site ou aplicação seja exibido corretamente em diferentes dispositivos, como desktops, tablets e smartphones.

\textbf{Fotogrametria}: Técnica que utiliza fotografias para realizar medições precisas de objetos ou ambientes, permitindo a reconstrução tridimensional (3D).

\textbf{Framework}: Estrutura de desenvolvimento de software que fornece bibliotecas, ferramentas e padrões para facilitar a criação de aplicações.

\textbf{Front-End}: Parte de um sistema ou aplicação que é visível e interativa para o usuário, envolvendo elementos como interface gráfica e navegação.

\textbf{GROQ}: Linguagem de consulta desenvolvida pelo Sanity para recuperar e manipular dados de forma eficiente em bancos NoSQL.

\textbf{Heurística de Nielsen}: Conjunto de princípios amplamente utilizados para avaliar a usabilidade de interfaces, baseado em diretrizes propostas por Jakob Nielsen.

\textbf{Imagem Responsiva}: Imagem que se ajusta automaticamente ao tamanho da tela do dispositivo, garantindo uma boa experiência visual em diferentes resoluções.

\textbf{JAMstack}: Arquitetura moderna de desenvolvimento \textit{web} que combina JavaScript, APIs e Markup para criar sites rápidos, seguros e escaláveis.

\textbf{Lighthouse}: Ferramenta de auditoria de desempenho, acessibilidade, SEO e outras métricas de qualidade para sites e aplicações \textit{web}.

\textbf{Malha Poligonal}: Representação 3D de um objeto ou ambiente composta por vértices, arestas e faces poligonais.

\textbf{Metahuman}: Tecnologia da Unreal Engine para criar avatares humanos altamente realistas, com detalhes como expressões faciais e texturas de pele.

\textbf{Modelagem 3D}: Processo de criar representações digitais tridimensionais de objetos ou ambientes utilizando softwares especializados.

\textbf{Next.js}: Framework React para desenvolvimento de aplicações \textit{web} modernas, com foco em desempenho e SEO otimizado.

\textbf{NoSQL}: Modelo de banco de dados que não segue a estrutura tradicional de tabelas relacionais, permitindo maior flexibilidade no armazenamento de dados.

\textbf{Nuvem de Pontos}: Conjunto de pontos no espaço 3D que representa a superfície de um objeto ou ambiente, frequentemente usado em fotogrametria.

\textbf{Patrimônio Virtual}: Representação digital de bens culturais, monumentos ou sítios históricos, criada para preservação e divulgação.

\textbf{Requisitos Funcionais}: Descrição das funções que um sistema deve executar, incluindo entradas, comportamentos e saídas esperadas.

\textbf{Requisitos Não Funcionais}: Características de qualidade que um sistema deve possuir, como desempenho, segurança, usabilidade e confiabilidade.

\textbf{Responsividade}: Capacidade de um site ou aplicação se adaptar a diferentes tamanhos de tela e dispositivos, garantindo uma boa experiência do usuário.

\textbf{Sanity CMS}: Plataforma de gerenciamento de conteúdo headless que utiliza um modelo NoSQL para armazenamento flexível de dados.

\textbf{Schema UI}: Biblioteca de componentes pré-construídos que facilita a criação de interfaces consistentes e personalizadas no Sanity.

\textbf{SEO}: Conjunto de técnicas para melhorar a visibilidade de um site nos resultados de mecanismos de busca, aumentando o tráfego orgânico.

\textbf{Texturização}: Processo de aplicar texturas e cores a modelos 3D para torná-los visualmente realistas.

\textbf{Unreal Engine}: Motor de criação de jogos e experiências interativas amplamente utilizado na indústria de entretenimento e simulações.

\textbf{Usabilidade}: Facilidade com que os usuários podem interagir com um sistema ou interface, realizando tarefas de forma eficiente e satisfatória.

\textbf{User Experience (UX)}: Experiência geral de um usuário ao interagir com um produto ou serviço, abrangendo aspectos como usabilidade, acessibilidade e emoções.

\textbf{URL}: é a sigla para "\textit{Uniform Resource Locator}", que significa "Localizador Uniforme de Recursos". É o endereço de um site ou arquivo na internet.

\textbf{Vercel}: Plataforma de hospedagem e deploy de aplicações \textit{web} modernas, com foco em desempenho e integração contínua.

\textbf{WCAG}: Diretrizes de Acessibilidade para Conteúdo Web, que estabelecem padrões globais para garantir que interfaces digitais sejam acessíveis a todos.

\textbf{Wireframe}: Esboço visual simplificado que representa a estrutura básica de uma interface, sem detalhes de design final.