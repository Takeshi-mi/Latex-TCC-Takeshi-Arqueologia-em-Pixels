\chapter{Método}
\label{Metodo}
\indent
Este capítulo oferece sugestões para descrever o método.

\section{O que devo escrever aqui?}%

O modelo 3D gerado passou por um processo de refinamento, que incluiu a remoção de ruídos, a aplicação de texturas para realçar a qualidade visual e garantir a qualidade do modelo final. O modelo 3D final foi fatiado em várias partes menores para então ser exportado no formato .fbx para a plataforma de desenvolvimento de jogos Unreal Engine 4.27 e colocado em escala Figura \ref{fig:unreal-print}, possibilitando a criação de um ambiente virtual imersivo, com recursos como simulação de luz, movimentações de câmera e personagens, além da criação de um ambiente virtual interativo para simular uma visita à Lapa da Pedra. Em seguida a simulação gerada na Unreal Engine foi empacotada para plataforma Windows e disponibilizada para download no site 
<endereco-do-site-aqui-quando-tiver>.


Cada ferramenta, tecnologia, definições e todo o arcabouço teórico do trabalho deve estar em seu \nameref{Referencial_Teorico}.
Aqui, no \nameref{Metodo} você deve escrever como utilizou as ferramentas, tecnologias e outros recursos para resolver o problema proposto com vistas a alcançar seus objetivos.
Este não é lugar para definir nada novo.


Não caia na tentação de dizer aqui os resultados! Guarde-os para o \nameref{Resultados}. Muitas vezes é interessante começar a escrita, justamente pelo \nameref{Resultados}.



Para finalizar, a qualidade do modelo 3D e do ambiente virtual foi avaliada visualmente, verificando a precisão das formas, a qualidade das texturas e a capacidade de navegação e interação dentro do ambiente virtual. O ambiente virtual 3D foi testado por usuários para avaliar a facilidade de navegação, a interação com os elementos do ambiente e a compreensão das informações apresentadas.


\section{Planejamento e Organização do Projeto}
\subsection{Kanban para Planejamento das Atividades}
Explicar como o Kanban foi usado para planejar e monitorar as tarefas.

\section{Coleta de Requisitos}
\subsection{Requisitos Funcionais}
Descrever as funcionalidades específicas do sistema.

\subsection{Requisitos Não Funcionais}
Listar requisitos de performance, segurança, etc.

\section{Modelagem do Sistema}
\subsection{Diagramas de Caso de Uso}
Desenvolver os cenários de uso para navegação e interação no ambiente virtual.

\subsubsection{Navegação no Ambiente Virtual}
Descrever como os usuários navegarão pelo ambiente virtual.

\subsubsection{Interação com o Banco de Dados}
Explicar como ocorrerá a interação com o banco de dados.

\subsubsection{Visualização de Artigos e Modelos}
Detalhar como os usuários visualizarão artigos e modelos 3D.

\subsection{Modelagem do Banco de Dados}
\subsubsection{MER}
Estruturar as entidades e relacionamentos.

\subsubsection{DER}
Descrever a estrutura física do banco.

\section{Desenvolvimento do Ambiente Virtual}
\subsection{Visitas ao Local e Captura de Imagens para Fotogrametria}
Descrever as visitas realizadas à Toca da Onça e o processo de captura de imagens com drones e câmeras para a fotogrametria. Especificar as condições do local, horários das visitas e configurações dos dispositivos utilizados.

\subsection{Processo de Fotogrametria na Toca da Onça}
Detalhar o processo de fotogrametria utilizado, especificando o equipamento e as técnicas adotadas para capturar as imagens necessárias à criação de um modelo 3D fiel ao local.

\subsection{Modelagem 3D e Unreal Engine}
Explicar as etapas de modelagem e importação do modelo 3D para o Unreal Engine, abordando a adequação e a otimização do modelo para o ambiente virtual.

\subsection{Metahumans e Representações Digitais}
Descrever o uso de Metahumans no ambiente virtual para a criação de experiências mais imersivas, explicando como as representações digitais enriquecem a experiência do usuário no ambiente.


\section{Desenvolvimento do Novo Site}
\subsection{Análise do Blog Arqueologia Formosa e seus problemas}
A avaliar o blog original e as melhorias propostas.
Será feita uma análise de UX com opiniões de 3 especialisas diferentes.
Conforme é feito nesse vídeo: 
Os analistas foram o Naoki Miura e o Victor Hugo Sales

\subsection{Prototipação com Figma}
Descrever o processo de prototipação.

\subsection{Implementação}
Detalhar o uso de HTML, CSS, Javascript e Laravel.

\subsection{Integração com Ambiente Virtual e Banco de Dados}
Explicitar o processo de integração.