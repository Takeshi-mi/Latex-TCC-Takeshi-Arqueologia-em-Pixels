\chapter{Metodologia}
\label{cap:metodologia}

Este capítulo descreve a metodologia adotada para o desenvolvimento do presente trabalho, contemplando o tipo de pesquisa, as etapas de construção do artefato e as técnicas utilizadas. A fundamentação teórica baseia-se nos princípios da \textit{Design Science Research} (DSR), conforme proposto por \cite{gregor2013positioning} e \citep{GregorHevner2013}, cuja abordagem propicia a criação de artefatos tecnológicos para solucionar problemas práticos e a contribuição para o conhecimento científico na área de Sistemas de Informação.

\section{Tipo de Pesquisa e Justificativa}
\label{sec:tipo_pesquisa}

Este estudo caracteriza-se como uma pesquisa aplicada, pois tem como objetivo resolver um problema real --- a necessidade de melhorar a divulgação e a preservação do sítio arqueológico Lapa da Pedra por meio de soluções digitais. A adoção da \textit{Design Science Research} é justificada pelo fato de que, além de propor um \textbf{artefato} (site JAMstack + ambiente virtual), busca-se avaliar e aperfeiçoar sua utilidade e eficácia, documentando todo o processo de desenvolvimento.

Segundo \citeonline{Hevner2004}, a DSR orienta-se pelas seguintes etapas: \textit{(i)} identificação do problema, \textit{(ii)} definição de objetivos, \textit{(iii)} desenho e desenvolvimento do artefato, \textit{(iv)} demonstração, \textit{(v)} avaliação e \textit{(vi)} comunicação. No presente projeto, essas etapas guiam as atividades desde a concepção do sistema até a sua verificação prática, garantindo rigor acadêmico e embasamento metodológico.

\section{Abordagem de Desenvolvimento}
\label{sec:abordagem_desenvolvimento}

Para gerenciar o ciclo de desenvolvimento de software, optou-se pela abordagem \textbf{Kanban}, que utiliza quadros de tarefas divididas em colunas (\textit{A Fazer}, \textit{Em Andamento}, \textit{Concluído}). Essa metodologia traz flexibilidade, pois possibilita a priorização dinâmica das atividades, além de oferecer transparência quanto ao progresso do projeto.

O gerenciamento de tarefas foi realizado no \textit{Notion}, que permite a criação de quadros de Kanban, atribuição de responsáveis e monitoramento do andamento das entregas. Já o controle de versão do código seguiu o fluxo de \textit{pull requests} no \textit{GitHub}, incluindo revisões de código e resolução de conflitos. Essa combinação de ferramentas contribuiu para a organização do trabalho e para a manutenção da qualidade das entregas.

\subsection{Cronograma de Atividades}
\label{subsec:cronograma_atividades}

A Tabela \ref{tab:cronograma} ilustra o cronograma geral do projeto, distribuído em marcos principais:

\begin{table}[H]
    \centering
    \caption{Cronograma de desenvolvimento}
    \label{tab:cronograma}
    \begin{tabular}{p{4.3cm} p{3.5cm} p{7cm}}
        \hline
        \textbf{Fase} & \textbf{Período} & \textbf{Principais Entregas} \\
        \hline
        Coleta de Requisitos       & Semana 1-2 & Definição de requisitos funcionais e não funcionais \\
        Modelagem de Dados         & Semana 2-3 & Diagrama conceitual de entidades e relacionamento \\
        Prototipagem (Figma)       & Semana 3-4 & Criação de \textit{wireframes} e protótipos das telas \\
        Implementação Web (Next.js + Sanity) & Semana 4-6 & Configuração do \textit{front-end}, rotas, integração com CMS \\
        Fotogrametria (Captura de Imagens) & Semana 5-6 & Obtenção de fotos aéreas e terrestres, processamento no Reality Capture \\
        Ambiente Virtual (Unreal)  & Semana 6-8 & Importação dos modelos 3D, criação de interações e configuração de iluminação \\
        Testes e Avaliação         & Semana 8-9 & Aplicação de testes de usabilidade, performance e ajustes finais \\
        Documentação e Conclusão   & Semana 9-10 & Finalização do TCC e revisão textual \\
        \hline
    \end{tabular}
\end{table}

\section{Coleta de Requisitos}
\label{sec:coleta_requisitos}

A coleta de requisitos baseou-se na análise do \textit{blog} anterior (manuscrito pelo professor Edson Borges) e em entrevistas informais com professores e alunos do Instituto Federal de Goiás (IFG). O levantamento contemplou tanto requisitos funcionais quanto não funcionais, os quais direcionaram a estrutura do site e a implementação do ambiente virtual. Os requisitos funcionais incluem:

\begin{itemize}
    \item Publicar e organizar artigos, relatórios e materiais acadêmicos relacionados à arqueologia formosense;
    \item Exibir galeria de imagens do sítio Lapa da Pedra;
    \item Oferecer recurso de busca por categorias (e-book, publicação, relatório, etc.);
    \item Disponibilizar modelos 3D para \textit{download} ou visualização \textit{online};
    \item Acesso a um ambiente virtual imersivo para análise das pinturas rupestres.
\end{itemize}

Já entre os requisitos não funcionais, destacam-se:
\begin{itemize}
    \item \textbf{Desempenho}: carregamento rápido em conexões padrão;
    \item \textbf{Acessibilidade}: conformidade com normas WCAG 2.1 (nível AA);
    \item \textbf{Segurança}: uso de HTTPS e prevenção contra ataques (XSS, CSRF);
    \item \textbf{Escalabilidade}: possibilidade de adicionar novos sítios arqueológicos sem reestruturar todo o sistema.
\end{itemize}

\section{Modelagem de Dados e Arquitetura}
\label{sec:modelagem_arquitetura}

Uma vez elucidados os requisitos, elaborou-se um diagrama conceitual para compreender as principais entidades e seus relacionamentos, mesmo com a adoção de um CMS \textit{headless} (Sanity), que opera em arquitetura NoSQL. Tal diagrama contribui para a organização lógica de documentos (p. ex., “Artigo”, “Sítio Arqueológico”, “Galeria de Imagens”) e facilita a manutenção do conteúdo ao longo do tempo.

Optou-se pela \textbf{arquitetura JAMstack} (JavaScript, APIs e Markup), na qual o \textit{frontend} em \textbf{Next.js} se comunica com o CMS \textbf{Sanity}, enquanto grandes arquivos (modelos 3D, fotos) são hospedados em serviços externos, como Flickr e Google Drive. O \textit{deploy} do site foi realizado na \textbf{Vercel}, explorando a geração de páginas estáticas (SSG) para maior desempenho.

\section{Critérios de Avaliação}
\label{sec:criterios_avaliacao}

Estabeleceram-se, por fim, os critérios e as métricas de avaliação do artefato:

\begin{enumerate}
    \item \textbf{Usabilidade}: verificação das \textit{heurísticas de Nielsen e Kaniasty} por meio da ferramenta \textit{Heurio}, coleta de feedback de usuários e tempo médio para realizar tarefas;
    \item \textbf{Desempenho}: medição do tempo de carregamento com o Google Lighthouse, bem como análise de responsividade em diferentes dispositivos;
    \item \textbf{Qualidade do Modelo 3D e Ambiente Virtual}: verificação da fidelidade das texturas, fluidez de navegação e experiência imersiva;
    \item \textbf{Acessibilidade}: checagem dos requisitos WCAG 2.1 (nível AA), como contraste de cores e navegação por teclado.
\end{enumerate}

Os resultados desses testes e avaliações serão discutidos no Capítulo \ref{resultados}, demonstrando se os objetivos propostos foram atendidos e em que medida o artefato se mostra eficaz na divulgação e preservação do sítio Lapa da Pedra.

\noindent\rule{12cm}{0.4pt}
