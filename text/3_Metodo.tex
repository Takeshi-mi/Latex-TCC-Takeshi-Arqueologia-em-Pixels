\chapter{Metodologia}
\label{cap:metodologia}

O presente trabalho adota a metodologia \gls{dsr}, conforme descrita por \cite{peffers2007design}, que propõe um modelo estruturado para a construção e avaliação de artefatos tecnológicos. A \gls{dsr} é amplamente utilizada em pesquisas de computação e engenharia de software, pois permite a criação de soluções inovadoras baseadas em problemas reais \citep{horita2019design}.

\section{Fases da Design Science Research}

\cite{peffers2007design} definem seis fases principais para a DSR:

\begin{enumerate}
    \item \textbf{Identificação do Problema e Motivação}: Definição do problema e justificativa de sua relevância.
    \item \textbf{Definição dos Objetivos da Solução}: Especificação do que a solução deve alcançar.
    \item \textbf{Design e Desenvolvimento}: Construção do artefato proposto dividida em ciclos.
    \item \textbf{Demonstração}: Aplicação da solução em um contexto real.
    \item \textbf{Avaliação}: Verificação da eficácia da solução.
    \item \textbf{Comunicação}: Divulgação dos resultados para a comunidade científica e profissionais da área.
\end{enumerate}

\section{Adaptação para Cinco Fases}

Neste trabalho, as fases da DSR foram adaptadas para um modelo de cinco etapas, de forma a organizar melhor o fluxo do projeto:

\begin{enumerate}
    \item \textbf{Identificação do Problema};
    \item \textbf{Definição dos Objetivos da Solução};
    \item \textbf{Design e Desenvolvimento};
    \item \textbf{Avaliação};
    \item \textbf{Comunicação}.
\end{enumerate}

Essa adaptação mantém a essência da DSR ao abranger a identificação do problema, definição dos objetivos, construção da solução, avaliação e comunicação dos resultados, garantindo uma abordagem estruturada e alinhada aos objetivos do projeto.

\section{Fase 1: Identificação do Problema}\label{definicao_do_problema}
O problema central foi a ineficácia do site antigo (Blogger) em divulgar e preservar o patrimônio arqueológico, devido a limitações técnicas, baixa usabilidade e identidade visual desatualizada. Tudo isso compromete o alcance e a experiência do usuário. Além disso, as abordagens tradicionais de divulgação digital, baseadas apenas em imagens e textos, não permitem uma exploração mais imersiva e interativa do patrimônio, reduzindo o engajamento e dificultando a compreensão espacial dos sítios arqueológicos. Portanto, objetiva-se a construção de um Ambiente Virtual 3D que possa ser baixado pelo site.

Para validá-lo, realizou-se:

\begin{itemize}
    \item \textbf{Visitas de campo}: Para coleta de imagens e informações sobre os sítios arqueológicos. Observou-se deterioração humana e natural, como pichações e erosão.
    \item \textbf{Análise Heurística}: Baseada nas 10 heurísticas de Nielsen, identificando falhas críticas (ex: navegação confusa).
\end{itemize}

\subsection{Visitas de Campo}

Detalhar as visitas, colocar fotos, descrever o que foi observado, etc.

\subsection{Análise Heurística do Antigo Site}
A análise heurística foi realizada no site antigo (Blogger) com base nas 10 heurísticas de Nielsen com o auxílio da ferramenta Heurio. A análise foi feita por quatro avaliadores: Naoki Rafael Miura, Victor Hugo Sales dos Reis, Sara Candido Fernandes e Erik Takeshi Miura, que identificaram problemas de usabilidade e navegabilidade.

Mostrar print do Heurio, descrever os problemas encontrados, etc.

\section{Fase 2: Definição dos Objetivos da Solução}\label{definicao_dos_objetivos_da_solucao}
\subsection{Coleta de Requisitos}
Coleta de requisitos foi realizada por meio de questionário online e entrevistas formais e informais com professores do IFG, resultando em 13 requisitos funcionais (Apêndice \ref{ap:requisitos-site-table}) e 13 não funcionais (Apêndice \ref{ap:requisitos-nao-funcionais-site-table}) para o site. Para o ambiente virtual há quatro requisitos funcionais (Apêndice \ref{ap:requisitos-ambiente-table}) e 2 requisitos não funcionais (Apêndice \ref{ap:requisitos-nao-funcionais-ambiente}). Todos os requisitos podem ser consultados no apêndice. 

\subsubsection{Identificação dos Requisitos}
\label{sec:identificacao_requisitos}

Por convenção e para facilitar a identificação dos casos de uso junto aos contextos, a referência é feita de acordo com o esquema abaixo:

\textbf{Sigla de subseção | numeração}

Os requisitos são identificados por uma sigla que representa a subseção (por exemplo, "RFS" para Requisitos Funcionais do site, “RNFA” para Requisitos Não Funcionais do Ambiente Virtual) seguida de um número sequencial.

\subsubsection{Prioridades dos Requisitos}
\label{sec:prioridades_requisitos}

Para estabelecer a prioridade dos requisitos, foram adotadas as denominações: essencial, importante e desejável. Abaixo temos a descrição de significado de cada uma dessas denominações:

\begin{table}[H]
\centering
\caption{Prioridades dos Requisitos}
\label{tab:prioridades_requisitos}
\begin{tabular}{|l|p{10cm}|}
\hline
\textbf{Prioridade} & \textbf{Descrição} \\ \hline
Essencial & Requisito sem o qual o sistema não entra em funcionamento. \\ \hline
Importante & Requisito que, sem ele, o sistema entra em funcionamento, mas de forma não satisfatória. \\ \hline
Desejável & Requisito que pode ser deixado para versões futuras sem comprometer as funcionalidades básicas. \\ \hline
\end{tabular}
\end{table}

\subsubsection{Requisitos Documentados}
\label{sec:requisitos_documentados}
{\small 
\begin{longtable}{|p{2.5cm}|p{4cm}|p{6cm}|p{2cm}|}
\caption{Exemplo da Tabela de Requisitos gerada}
\label{table:exemplo_tabela_requisitos} \\
\hline
\textbf{Identificador} & \textbf{Nome} & \textbf{Descrição} & \textbf{Prioridade} \\
\hline
\endfirsthead
\hline
\textbf{Identificador} & \textbf{Nome} & \textbf{Descrição} & \textbf{Prioridade} \\
\hline
\endhead
RFS001 & Exemplo de Nome & Exemplo de descrição do requisito & Essencial, Importante ou Desejável \\ \hline
\end{longtable}
}
Para acessar os requisitos gerados na íntegra, consulte o Apêndice \ref{apendice}.


\section{Fase 3: Design e Desenvolvimento}\label{sec: Fase 3 design e desenvolvimento}
Após a identificação do problema \ref{definicao_do_problema} e a definição dos objetivos \ref{definicao_dos_objetivos_da_solucao} da solução, esta seção descreve o processo de design e desenvolvimento do projeto. Seguindo a abordagem da \gls{dsr}, conforme descrita por \cite{peffers2007design}, o desenvolvimento foi estruturado em ciclos iterativos, permitindo refinamento contínuo dos artefatos conforme novas descobertas e avaliações eram feitas.
Neste projeto, os ciclos foram organizados de forma a contemplar as principais etapas do desenvolvimento da solução:

% Ciclo 1
\subsection*{Ciclo 1: Fotogrametria e Construção do Modelo 3D} \label{subsec:ciclo1}

\subsubsection*{Objetivo do Ciclo}
Criar um modelo 3D detalhado baseado em imagens capturadas em campo. Atender ao RFA001 (Modelo 3D de Alta Qualidade) e garantir compatibilidade com Unreal Engine 5.4. 

\subsubsection*{Ações Realizadas}
\begin{enumerate}
    \item Captura de imagens de alta resolução de diferentes ângulos do sítio da Lapa da Pedra e seleção para remover fotos com ruídos.
    \item Processamento no Reality Capture para reconstrução fotogramétrica, resultando em um modelo inicial com 102 milhões de polígonos.
    \item Simplificação do modelo para 5 milhões de polígonos, garantindo compatibilidade com a Unreal Engine (RFA001).
    \item Reprojeção de textura do modelo de alta resolução para o modelo simplificado, mantendo a qualidade visual.
\end{enumerate}

\textbf{Resultado}: Modelo 3D detalhado e otimizado para uso no ambiente virtual, garantindo fidelidade visual e desempenho adequado na Unreal Engine 5.4 (RFA001).

% Ciclo 2
\subsection*{Ciclo 2: Construção do Ambiente Virtual na Unreal Engine 5.4} \label{subsec:ciclo2}

\subsubsection*{Objetivo do Ciclo}
Atender aos requisitos RFA002 (Navegação em Terceira Pessoa), RFA003 (Alternância de Câmera), RFA004 (Avatar Personalizado), RNFA001 (Compatibilidade com Windows), RNFA002 (Documentação de Instalação) e RNFA003 (Instalação Simplificada).

\subsubsection*{Ações Realizadas}
\begin{enumerate}
    \item Importação do modelo 3D otimizado na Unreal Engine 5.4.
    \item Criação do cenário virtual com texturas realistas e iluminação dinâmica.
    \item Implementação do sistema de navegação em terceira pessoa com controle de um avatar interativo (RFA002).
    \item Criação de um avatar personalizado no Metahuman com aparência semelhante ao professor Edson Borges (RFA004).
    \item Desenvolvimento de um sistema de alternância de câmera, permitindo troca entre visão em primeira e terceira pessoa (RFA003).
    \item Configuração de otimizações gráficas para garantir um bom desempenho em máquinas com Windows 10 ou superior (RNFA001).
    \item Empacotamento do ambiente virtual em um arquivo executável (.exe) para distribuição, garantindo instalação simplificada para usuários finais (RNFA003).
    \item Desenvolvimento de um guia de instalação e configuração para orientar os usuários na utilização do ambiente virtual (RNFA002).
\end{enumerate}

\textbf{Resultado}: Ambiente virtual interativo e otimizado, permitindo exploração em terceira pessoa com opção de alternância para visão em primeira pessoa, avatar personalizado, compatibilidade com Windows 10+ e um instalador simplificado para facilitar a distribuição e configuração pelos usuários finais (RFA002, RFA003, RFA004, RNFA001, RNFA002, RNFA003).

% Ciclo 3
\subsection*{Ciclo 3: Prototipação do Novo Site} \label{subsec:ciclo3}

\subsubsection*{Objetivo do Ciclo}
Atender aos requisitos RNFS006 (Acessibilidade), RNFS008 (Responsividade) e RNFS014 (Usabilidade).

\subsubsection*{Ações Realizadas}
\begin{enumerate}
    \item Análise das referências e inspirações enviadas pelo professor Edson Borges por meio de um questionário, garantindo alinhamento com suas expectativas e visão para o site.
    \item Reorganização da arquitetura da informação, agrupando conteúdos relevantes, reduzindo informações desnecessárias e minimizando a quantidade de cliques para que o usuário encontrasse a informação desejada (RF006, RNFS014).
    \item Criação de wireframes para estruturar a interface e definir a disposição dos elementos visuais e interativos do site.
    \item Definição da paleta de cores e tipografia, garantindo coerência visual e acessibilidade (RNFS006, RNFS014).
    \item Criação de uma nova logo, inspirada nas cores das artes rupestres encontradas na Lapa da Pedra, reforçando a identidade visual do projeto.
    \item Desenvolvimento de um protótipo de alta fidelidade no Figma, representando a versão final do site com design refinado (RF002, RF006, RNFS014).
    \item Validação e refinamento do protótipo a partir de feedbacks coletados com o professor e colegas, garantindo alinhamento com as necessidades do projeto e uma experiência de usuário otimizada.
    \item Organização eficiente do projeto no Figma garantindo um ambiente estruturado e facilitando futuras iterações e implementações.
\end{enumerate}

\textbf{Resultado}: Protótipo de alta fidelidade bem estruturado e validado, garantindo acessibilidade, melhor navegação, identidade visual consistente e um layout flexível para futuras alterações e desenvolvimento (RNFS006, RNFS008, RNFS014).
% 4
\subsection*{Ciclo 4: Desenvolvimento do Novo Site} \label{subsec:ciclo4}

\subsubsection*{Objetivo do Ciclo}
Atender aos requisitos funcionais e não funcionais do site. O desenvolvimento abrange a maioria dos requisitos especificados nas tabelas \ref{ap:requisitos-site-table} e \ref{ap:requisitos-nao-funcionais-site-table}, garantindo acessibilidade, responsividade, gerenciamento dinâmico de conteúdo e otimização de desempenho.

\subsubsection*{Ações Realizadas}
\begin{enumerate}
\item \textbf{Configuração do ambiente de desenvolvimento}: Definição da arquitetura \textit{JAMstack}, utilizando o \textit{framework} \textit{Next.js} e a biblioteca \textit{Schema UI} no \textit{frontend} e \textit{Sanity} como CMS headless, além da configuração do repositório no Git e escolha da hospedagem na Vercel para otimização de desempenho e redução de custos (RNFS001, RNFS008, RNFS011).

    
    \item \textbf{Implementação da estrutura e funcionalidades do site}
    \begin{enumerate}
        \item Implementação de páginas-chave:
        \begin{enumerate}
            \item Página inicial com navegação hierárquica
            \item Seções específicas para sítios arqueológicos
            \item Página de Contato
            \item Página de trabalhos escritos com filtros
        \end{enumerate}
    \end{enumerate}
    
\end{enumerate}

\section{Fase 4: Avaliação} \label{sec:avaliacao-dsr}

    \textbf{Validação Técnica}
    \begin{enumerate}
        \item Testes de performance com Lighthouse;
        \item Verificação de compatibilidade em 6 navegadores diferentes;
        \item Avaliação heurística do novo site e \textit{feedbacks}.
    \end{enumerate}
    

\section{Fase 5: Comunicação} \label{sec:comunicacao-dsr}

Os resultados foram divulgados através de:
\begin{itemize}
    \item \textbf{Documentação Acadêmica}: Este TCC
    \item \textbf{Publicação Online}: Site em \url{https://arqueologiaformosa.com.br}.
\end{itemize}


