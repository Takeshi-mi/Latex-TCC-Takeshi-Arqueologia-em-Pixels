\chapter{Metodologia}
\label{cap:metodologia}

Este capítulo descreve a metodologia adotada para o desenvolvimento do presente trabalho, contemplando o tipo de pesquisa, as etapas de construção do artefato e as técnicas utilizadas. A fundamentação teórica baseia-se nos princípios da \textit{Design Science Research} (DSR), conforme proposto por \cite{gregor2013positioning} e , cuja abordagem propicia a criação de artefatos tecnológicos para solucionar problemas práticos e a contribuição para o conhecimento científico na área de Sistemas de Informação.

\section{Tipo de Pesquisa e Justificativa}
\label{sec:tipo_pesquisa}

Este estudo caracteriza-se como uma pesquisa aplicada, pois tem como objetivo resolver um problema real --- a necessidade de melhorar a divulgação e a preservação do sítio arqueológico Lapa da Pedra por meio de soluções digitais. A adoção da \textit{Design Science Research} é justificada pelo fato de que, além de propor um \textbf{artefato} (site JAMstack + ambiente virtual), busca-se avaliar e aperfeiçoar sua utilidade e eficácia, documentando todo o processo de desenvolvimento.

Segundo \cite{gregor2013positioning}, a DSR orienta-se pelas seguintes etapas: \textit{(i)} identificação do problema, \textit{(ii)} definição de objetivos, \textit{(iii)} desenho e desenvolvimento do artefato, \textit{(iv)} demonstração, \textit{(v)} avaliação e \textit{(vi)} comunicação. No presente projeto, essas etapas guiam as atividades desde a concepção do sistema até a sua verificação prática, garantindo rigor acadêmico e embasamento metodológico.

\section{Abordagem de Desenvolvimento}
\label{sec:abordagem_desenvolvimento}

Para gerenciar o ciclo de desenvolvimento de software, optou-se pela abordagem \textbf{Kanban}, que utiliza quadros de tarefas divididas em colunas (\textit{A Fazer}, \textit{Em Andamento}, \textit{Concluído}). Essa metodologia traz flexibilidade, pois possibilita a priorização dinâmica das atividades, além de oferecer transparência quanto ao progresso do projeto.

O gerenciamento de tarefas foi realizado no \textit{Notion}, que permite a criação de quadros de Kanban, atribuição de responsáveis e monitoramento do andamento das entregas. Já o controle de versão do código seguiu o fluxo de \textit{pull requests} no \textit{GitHub}, incluindo revisões de código e resolução de conflitos. Essa combinação de ferramentas contribuiu para a organização do trabalho e para a manutenção da qualidade das entregas.

\subsection{Cronograma de Atividades}
\label{subsec:cronograma_atividades}

A Tabela \ref{tab:cronograma} ilustra o cronograma geral do projeto, distribuído em marcos principais:

\begin{table}[H]
    \centering
    \caption{Cronograma de desenvolvimento}
    \label{tab:cronograma}
    \begin{tabular}{p{4.3cm} p{3.5cm} p{7cm}}
        \hline
        \textbf{Fase} & \textbf{Período} & \textbf{Principais Entregas} \\
        \hline
        Coleta de Requisitos       & Semana 1-2 & Definição de requisitos funcionais e não funcionais \\
        Modelagem de Dados         & Semana 2-3 & Diagrama conceitual de entidades e relacionamento \\
        Prototipagem (Figma)       & Semana 3-4 & Criação de \textit{wireframes} e protótipos das telas \\
        Implementação Web (Next.js + Sanity) & Semana 4-6 & Configuração do \textit{front-end}, rotas, integração com CMS \\
        Fotogrametria (Captura de Imagens) & Semana 5-6 & Obtenção de fotos aéreas e terrestres, processamento no Reality Capture \\
        Ambiente Virtual (Unreal)  & Semana 6-8 & Importação dos modelos 3D, criação de interações e configuração de iluminação \\
        Testes e Avaliação         & Semana 8-9 & Aplicação de testes de usabilidade, performance e ajustes finais \\
        Documentação e Conclusão   & Semana 9-10 & Finalização do TCC e revisão textual \\
        \hline
    \end{tabular}
\end{table}

\section{Coleta de Requisitos}
\label{sec:coleta_requisitos}

A coleta de requisitos baseou-se na análise do \textit{blog} anterior (manuscrito pelo professor Edson Borges) e em entrevistas informais com professores e alunos do Instituto Federal de Goiás (IFG). O levantamento contemplou tanto requisitos funcionais quanto não funcionais, os quais direcionaram a estrutura do site e a implementação do ambiente virtual. Os requisitos funcionais incluem:

\begin{itemize}
    \item Publicar e organizar artigos, relatórios e materiais acadêmicos relacionados à arqueologia formosense;
    \item Exibir galeria de imagens do sítio Lapa da Pedra;
    \item Oferecer recurso de busca por categorias (e-book, publicação, relatório, etc.);
    \item Disponibilizar modelos 3D para \textit{download} ou visualização \textit{online};
    \item Acesso a um ambiente virtual imersivo para análise das pinturas rupestres.
\end{itemize}

Já entre os requisitos não funcionais, destacam-se:
\begin{itemize}
    \item \textbf{Desempenho}: carregamento rápido em conexões padrão;
    \item \textbf{Acessibilidade}: conformidade com normas WCAG 2.1 (nível AA);
    \item \textbf{Segurança}: uso de HTTPS e prevenção contra ataques (XSS, CSRF);
    \item \textbf{Escalabilidade}: possibilidade de adicionar novos sítios arqueológicos sem reestruturar todo o sistema.
\end{itemize}

\section{Identificação dos Requisitos}
\label{sec:identificacao_requisitos}

Por convenção e para facilitar a identificação dos casos de uso junto aos contextos, a referência é feita de acordo com o esquema abaixo:

\textbf{\[ Sigla de subseção | numeração\]}

Os requisitos são identificados por uma sigla que representa a subseção (por exemplo, "RF" para Requisitos Funcionais, “RNF” para Requisitos Não Funcionais) seguida de um número sequencial.

\section{Prioridades dos Requisitos}
\label{sec:prioridades_requisitos}

Para estabelecer a prioridade dos requisitos, foram adotadas as denominações: essencial, importante e desejável. Abaixo temos a descrição de significado de cada uma dessas denominações:

\begin{table}[H]
\centering
\caption{Prioridades dos Requisitos}
\label{tab:prioridades_requisitos}
\begin{tabular}{|l|p{10cm}|}
\hline
\textbf{Prioridade} & \textbf{Descrição} \\ \hline
Essencial & É o requisito sem o qual o sistema não entra em funcionamento. Requisitos essenciais são requisitos imprescindíveis, que têm que ser implementados impreterivelmente. \\ \hline
Importante & É o requisito sem o qual o sistema entra em funcionamento, mas de forma não satisfatória. Requisitos importantes devem ser implementados, mas, se não forem, o sistema poderá ser implantado e usado mesmo assim. \\ \hline
Desejável & É o requisito que não compromete as funcionalidades básicas do sistema, isto é, o sistema pode funcionar de forma satisfatória sem ele. Requisitos desejáveis são requisitos que podem ser deixados para versões posteriores do sistema, caso não haja tempo hábil para implementá-los na versão que está sendo especificada. \\ \hline
\end{tabular}
\end{table}

\section{Descrição dos Requisitos}
\label{sec:descricao_requisitos}

\subsection{Requisitos Funcionais}

{\small % Reduz o tamanho da fonte para economizar espaço
\begin{longtable}{|>{\raggedright}p{2.5cm}|>{\raggedright}p{4cm}|p{6cm}|>{\raggedright}p{2cm}|}
\caption{Tabela de Requisitos Funcionais}
\label{tab:requisitos_funcionais}
\hline
\textbf{Identificador} & \textbf{Nome} & \textbf{Descrição} & \textbf{Prioridade} \\
\hline
\endfirsthead % Cabeçalho da primeira página
\hline
\textbf{Identificador} & \textbf{Nome} & \textbf{Descrição} & \textbf{Prioridade} \\
\hline
\endhead % Cabeçalho das páginas seguintes
RF001 & Publicar Trabalhos Escritos & Permitir a publicação de artigos, relatórios de PIBIC, e-books e outras produções acadêmicas, organizados por categorias e tags. & Essencial \\
\hline
RF002 & Exibir Galeria de Imagens & Oferecer uma galeria de imagens relacionada aos sítios arqueológicos, com suporte para visualização ampliada. & Importante \\
\hline
RF003 & Baixar Modelos 3D & Oferecer a opção de download de modelos 3D do sítio arqueológico. & Essencial \\
\hline
RF004 & Ambiente Virtual para Download & Disponibilizar o arquivo executável do ambiente virtual desenvolvido na Unreal Engine para download pelos usuários. & Importante \\
\hline
RF005 & Compartilhar nas Redes Sociais & Permitir o compartilhamento de conteúdos e links diretamente nas redes sociais. & Desejável \\
\hline
RF006 & Buscar Trabalhos por Filtros & Oferecer um mecanismo de busca que permita filtrar trabalhos escritos por categorias e por sítios arqueológicos. & Essencial \\
\hline
RF007 & Formulário de Contato & Disponibilizar um formulário para que os usuários possam entrar em contato com os administradores do sistema. & Importante \\
\hline
RF008 & Autenticação de Usuário & Implementar autenticação de usuário, diferenciando permissões entre administradores e leitores. & Essencial \\
\hline
RF009 & Gerenciamento de Conteúdo & Permitir que administradores adicionem, editem ou removam postagens e outros conteúdos diretamente no sistema. & Essencial \\
\hline
RF010 & Alterar Estrutura do Site & Administradores devem poder modificar a estrutura do site, como layout e menus, sem a necessidade de programação. & Desejável \\
\hline
RF011 & Modo Escuro/Claro & Oferecer uma alternância entre modos escuro e claro para a interface. & Desejável \\
\hline
RF012 & Exibir Mapa dos Sítios Arqueológicos & Exibir mapas interativos, integrados ao Google Maps, para a localização dos sítios arqueológicos. & Importante \\
\hline
RF013 & Autenticação via Google & Permitir que usuários façam login utilizando suas contas Google para acessar áreas restritas. & Essencial \\
\hline
RF014 & Monitoramento de Acessos & Integrar ferramentas como Google Analytics para coletar dados sobre o número de acessos e comportamento dos usuários no site. & Importante \\
\hline
\end{longtable}
} % Fim do ambiente \small

\section{Requisitos Não Funcionais}

{\small % Reduz o tamanho da fonte para economizar espaço
\begin{longtable}{|>{\raggedright}p{2.5cm}|>{\raggedright}p{4cm}|p{6cm}|>{\raggedright}p{2cm}|}
\caption{Tabela de Requisitos Não Funcionais}
\label{tab:requisitos__nao_funcionais}
\hline
\textbf{Identificador} & \textbf{Nome} & \textbf{Descrição} & \textbf{Prioridade} \\
\hline
\endfirsthead % Cabeçalho da primeira página
\hline
\textbf{Identificador} & \textbf{Nome} & \textbf{Descrição} & \textbf{Prioridade} \\
\hline
\endhead % Cabeçalho das páginas seguintes
RNF001 & Tempo de Carregamento & O sistema deve carregar as páginas em até 3 segundos em uma conexão de 5 Mbps. & Essencial \\
\hline
RNF002 & Performance Visível & As páginas principais devem exibir conteúdo visível para o usuário em até 1 segundo após o carregamento inicial. & Essencial \\
\hline
RNF003 & Escalabilidade & O sistema deve suportar um aumento de até 10 vezes no número de visitantes mensais sem comprometer o desempenho. & Importante \\
\hline
RNF004 & Manutenibilidade & O código do sistema deve ser modular e seguir boas práticas de desenvolvimento, facilitando atualizações e correções futuras. & Importante \\
\hline
RNF005 & Segurança & O sistema deve garantir a segurança dos dados por meio de HTTPS, além de implementar medidas contra ataques comuns, como XSS e CSRF. & Essencial \\
\hline
RNF006 & Acessibilidade & O site deve seguir as diretrizes WCAG 2.1 (nível AA), garantindo suporte a leitores de tela, navegação por teclado e contraste adequado. & Importante \\
\hline
RNF007 & Compatibilidade com Navegadores & O sistema deve ser compatível com os navegadores mais utilizados (Google Chrome, Firefox, Safari e Edge) em suas versões mais recentes. & Importante \\
\hline
RNF008 & Responsividade & O site deve ser responsivo, oferecendo uma experiência adequada em dispositivos móveis, tablets e desktops. & Essencial \\
\hline
RNF009 & Confiabilidade & O sistema deve ter uma disponibilidade mínima de 99,9\% mensal. & Essencial \\
\hline
RNF010 & Tolerância a Falhas & O sistema deve implementar backups automáticos diários para garantir a recuperação de dados em caso de falhas. & Importante \\
\hline
RNF011 & Custos Operacionais Reduzidos & O sistema deve ser projetado para funcionar com ferramentas e serviços gratuitos sempre que possível, reduzindo custos operacionais. & Desejável \\
\hline
RNF012 & SEO (Otimização para Motores de Busca) & O sistema deve ser otimizado para SEO, garantindo que o conteúdo seja facilmente encontrado em mecanismos de busca como Google. & Importante \\
\hline
RNF013 & Integração com Serviços Externos & O sistema deve permitir integração fluida com plataformas como Sanity (CMS), Google Maps, Flickr e Google Drive. & Essencial \\
\hline
\end{longtable}
} % Fim do ambiente \small


\section{Modelagem de Dados e Arquitetura}
\label{sec:modelagem_arquitetura}

Uma vez elucidados os requisitos, elaborou-se um diagrama conceitual para compreender as principais entidades e seus relacionamentos, mesmo com a adoção de um CMS \textit{headless} (Sanity), que opera em arquitetura NoSQL. Tal diagrama contribui para a organização lógica de documentos (p. ex., “Artigo”, “Sítio Arqueológico”, “Galeria de Imagens”) e facilita a manutenção do conteúdo ao longo do tempo.

Optou-se pela \textbf{arquitetura JAMstack} (JavaScript, APIs e Markup), na qual o \textit{frontend} em \textbf{Next.js} se comunica com o CMS \textbf{Sanity}, enquanto grandes arquivos (modelos 3D, fotos) são hospedados em serviços externos, como Flickr e Google Drive. O \textit{deploy} do site foi realizado na \textbf{Vercel}, explorando a geração de páginas estáticas (SSG) para maior desempenho.

\section{Critérios de Avaliação}
\label{sec:criterios_avaliacao}

Estabeleceram-se, por fim, os critérios e as métricas de avaliação do artefato:

\begin{enumerate}
    \item \textbf{Usabilidade}: verificação das \textit{heurísticas de Nielsen e Kaniasty} por meio da ferramenta \textit{Heurio}, coleta de feedback de usuários e tempo médio para realizar tarefas;
    \item \textbf{Desempenho}: medição do tempo de carregamento com o Google Lighthouse, bem como análise de responsividade em diferentes dispositivos;
    \item \textbf{Qualidade do Modelo 3D e Ambiente Virtual}: verificação da fidelidade das texturas, fluidez de navegação e experiência imersiva;
    \item \textbf{Acessibilidade}: checagem dos requisitos WCAG 2.1 (nível AA), como contraste de cores e navegação por teclado.
\end{enumerate}

Os resultados desses testes e avaliações serão discutidos no Capítulo \ref{resultados}, demonstrando se os objetivos propostos foram atendidos e em que medida o artefato se mostra eficaz na divulgação e preservação do sítio Lapa da Pedra.

\noindent\rule{12cm}{0.4pt}
