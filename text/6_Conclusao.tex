\chapter{Conclusão} \label{cap:conclusao}

Este trabalho demonstrou como tecnologias digitais inovadoras podem ser aplicadas à preservação e divulgação do patrimônio arqueológico, especificamente das pinturas rupestres do sítio Lapa da Pedra, em Formosa-GO. A seguir, são apresentadas as considerações finais, limitações encontradas e sugestões para trabalhos futuros.

\section{Considerações Finais} \label{sec:consideracoes}

Este trabalho cumpriu seus objetivos ao desenvolver uma plataforma digital que combina um site moderno e um ambiente virtual imersivo para a Lapa da Pedra. Conforme demonstrado nos Resultados (Seção \ref{cap:resultados}):
\begin{itemize}
    \item O novo site superou o anterior em usabilidade, desempenho e acessibilidade.
    \item O ambiente virtual permitiu uma exploração detalhada das pinturas rupestres, mesmo para usuários remotos.
\end{itemize}

A implementação deste trabalho não apenas cumpriu o objetivo proposto, mas também abriu portas para novas formas de educação patrimonial.

\section{Limitações e Dificuldades}
\label{sec:limitações}
Durante o desenvolvimento, algumas dificuldades foram enfrentadas:
\begin{itemize}
    \item Dificuldades na otimização dos modelos 3D para uso na Unreal Engine.
    \item Limitação de tempo para implementar todas as funcionalidades desejadas.
    \item Restrições no orçamento.
    \item Limitações hardware para produzir os modelos 3D e o ambiente virtual na Unreal. Houveram processos que demoraram mais de 30 horas para renderizar.
    \item Compatibilidade: O ambiente virtual não foi testado em hardware muito antigo, restringindo o acesso a usuários com computadores básicos.
\end{itemize}

\section{Trabalhos Futuros} \label{sec:futuro}

Para expandir o projeto, sugere-se:

\begin{itemize}
    \item Integração com Realidade Virtual (VR): Adicionar suporte a headsets como Oculus Rift ou Meta Quest, permitindo imersão total no ambiente virtual.
    \item Versão Mobile: Desenvolver um aplicativo para Android/iOS com recursos simplificados de visualização 3D.
    \item Novos Sítios Arqueológicos: Incorporar modelos 3D de outros sítios, como o Bisnau, usando a mesma infraestrutura.
    \item Chatbot com IA: Implementar IA no site para conversar com os usuários sobre os sítios e o patrimônio arqueológico.
    \item Implementação dos filtros e barra de busca para buscar os trabalhos escritos publicados por categoria, tipo ou sítio arqueológico.
    \item Envolvimento Comunitário: Workshops para capacitar moradores de Formosa-GO na documentação digital do patrimônio local.
\end{itemize}


\subsection{Contribuição para a Área}
Este trabalho contribuiu para a interseção entre arqueologia e tecnologia ao:
\begin{itemize}
    \item Oferecer um modelo replicável para digitalização de sítios arqueológicos em regiões subrepresentadas.
    \item Promover a conscientização sobre a importância da preservação cultural através de ferramentas acessíveis.
\end{itemize}

\subsection{Palavras Finais}
A preservação do patrimônio arqueológico é um desafio contínuo, mas projetos como este demonstram que a tecnologia pode ser uma aliada poderosa. Espera-se que esta iniciativa inspire novas pesquisas e políticas públicas voltadas à proteção digital de nossa herança cultural.