\chapter{Conclusão}
\label{Conclusao}

Esta é a conclusão do trabalho. 
Aqui são mostradas as contribuições para a ciência ou para a área em que se aplica a solução desenvolvida.
Também é importante mostrar os limites da contribuição e como eles podem ser rompidos em trabalhos futuros.

\section{Considerações Finais}
\label{sec:consideracoes_finais}
Avaliação do cumprimento dos objetivos e impacto do projeto na preservação e divulgação da arte rupestre.


\section{Limitações e Dificuldades}
\label{sec:limitações}
Durante o desenvolvimento, algumas dificuldades foram enfrentadas:
\begin{itemize}
    \item Dificuldades na otimização dos modelos 3D para uso na Unreal Engine.
    \item Limitação de tempo para implementar todas as funcionalidades desejadas.
    \item Restrições no orçamento.
    \item Limitações hardware para produzir os modelos 3D e o ambiente virtual na Unreal. Houveram processos que demoraram mais de 30 horas para renderizar.
\end{itemize}

\section{Trabalhos Futuros}
\label{sec:trabalhos_futuros}
Sugestões para expansão do projeto, como suporte a novos sítios arqueológicos, versão mobile e integração com VR.
