% Glossário para o TCC
\newglossaryentry{algoritmo}{
    name={Algoritmo},
    description={Sequência finita de instruções bem definidas para resolver um problema ou realizar uma tarefa}
}

\newglossaryentry{compilador}{
    name={Compilador},
    description={Software que traduz código-fonte de uma linguagem de programação para linguagem de máquina}
}

\newglossaryentry{backend}{
    name={Back-End},
    description={Parte do software que opera no servidor e gerencia lógica, banco de dados e APIs}
}

\newglossaryentry{frontend}{
    name={Front-End},
    description={Parte do software que interage diretamente com o usuário por meio de interfaces gráficas}
}

\newglossaryentry{api}{
    name={API},
    description={\textit{Application Programming Interface}, conjunto de rotinas e padrões para comunicação entre sistemas}
}

\newglossaryentry{rest}{
    name={REST},
    description={\textit{Representational State Transfer}, padrão de arquitetura para comunicação entre sistemas usando HTTP}
}

\newglossaryentry{unrealengine}{
    name={Unreal Engine},
    description={Plataforma de desenvolvimento de jogos e ambientes virtuais, amplamente usada para criação de experiências imersivas}
}

\newglossaryentry{fotogrametria}{
    name={Fotogrametria},
    description={Técnica que utiliza fotografias para medir distâncias, criar mapas ou modelos tridimensionais}
}

\newglossaryentry{ux-gloss}{
    name={UX},
    description={\textit{User Experience}, conjunto de práticas que visam melhorar a experiência do usuário ao interagir com um sistema ou produto}
}

\newglossaryentry{ui}{
    name={UI},
    description={\textit{User Interface}, design e estrutura visual de uma interface com foco na interação do usuário}
}



\newglossaryentry{nosql}{
    name={NoSQL},
    description={\textit{Not Only SQL}, categoria de sistemas de gerenciamento de banco de dados que não utilizam modelo relacional tradicional}
}

\newglossaryentry{inteligenciaartificial}{
    name={Inteligência Artificial},
    description={Área da ciência da computação que desenvolve sistemas capazes de realizar tarefas que normalmente requerem inteligência humana}
}

\newglossaryentry{sanity}{
    name={Sanity},
    description={Plataforma de gerenciamento de conteúdo (\textit{CMS}) que permite criar, armazenar e gerenciar conteúdo estruturado}
}

\newglossaryentry{nextjs}{
    name={Next.js},
    description={Framework baseado em React para desenvolvimento de aplicações web modernas com renderização no lado do servidor}
}

\newglossaryentry{fotorealismo}{
    name={Fotorealismo},
    description={Técnica de renderização que busca simular imagens extremamente próximas da realidade em ambientes virtuais}
}


\newglossaryentry{github}{
    name={GitHub},
    description={Plataforma para hospedagem de código-fonte e controle de versão, permitindo colaboração entre desenvolvedores}
}

\newglossaryentry{json}{
    name={JSON},
    description={\textit{JavaScript Object Notation}, formato leve para troca de dados estruturados}
}


% Exemplo de entradas para o glossário
\newglossaryentry{deploy}{
  name={Deploy},
  description={Processo de publicar ou atualizar uma aplicação em um ambiente de produção, tornando-a acessível aos usuários finais.}
}

\newglossaryentry{cdn}{
  name={CDN},
  description={Content Delivery Network (Rede de Distribuição de Conteúdo), uma infraestrutura de servidores distribuídos globalmente que acelera a entrega de conteúdo web ao reduzir a latência.}
}

\newglossaryentry{framework}{
  name={framework},
  description={Conjunto de ferramentas, bibliotecas e convenções que facilitam o desenvolvimento de software, fornecendo uma estrutura base para aplicações.}
}

\newglossaryentry{kanban}{
  name={Kanban},
  description={Método visual de gerenciamento de tarefas que utiliza colunas para representar o fluxo de trabalho, como "A fazer", "Em progresso" e "Concluído".}
}

\newglossaryentry{repositorio-git}{
  name={Repositório Git},
  description={Um local onde o código-fonte de um projeto é armazenado e versionado, permitindo o controle de alterações e colaboração entre desenvolvedores.}
}

\newglossaryentry{escalabilidade}{
  name={Escalabilidade},
  description={Capacidade de um sistema de lidar com o aumento de carga ou demanda sem comprometer o desempenho ou a disponibilidade.}
}

\newglossaryentry{dsr}{
    name={Design Science Research (DSR)},
    description={Metodologia de pesquisa baseada na construção e avaliação de artefatos para resolver problemas reais, amplamente usada na computação e engenharia de software.}
}