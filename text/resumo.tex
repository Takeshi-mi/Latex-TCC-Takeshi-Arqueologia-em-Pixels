Este trabalho de conclusão de curso apresenta o desenvolvimento de uma plataforma digital para preservação e divulgação do patrimônio arqueológico da região de Formosa, Goiás. O projeto consiste na criação de um site moderno e responsivo, aliado a um ambiente virtual tridimensional imersivo que permite a exploração das pinturas rupestres do sítio Lapa da Pedra. A metodologia adotada foi baseada no Design Science Research (DSR), estruturada em ciclos iterativos de desenvolvimento. Para o ambiente virtual, foram utilizadas técnicas de fotogrametria e modelagem 3D integradas à Unreal Engine 5.4, enquanto o site foi desenvolvido utilizando tecnologias modernas como Next.js e Sanity CMS. Os resultados demonstraram um avanço significativo na forma de explorar e preservar digitalmente o patrimônio arqueológico por meio do ambiente virtual, aliado às significativas melhorias em relação a forma como vinha sendo apresentado, atendendo plenamente aos requisitos de usabilidade, acessibilidade e desempenho estabelecidos. O projeto contribui para a educação patrimonial e preservação digital do patrimônio cultural, servindo como modelo replicável para outros sítios arqueológicos.

\begin{keywords}
Arqueologia Digital, Fotogrametria, Lapa da Pedra (Toca da Onça), Patrimônio Virtual, Programação Web JAMstack.
\end{keywords}

% Palavras-chave: Arqueologia Digital, Preservação Virtual, Patrimônio Cultural, Ambiente Virtual, Fotogrametria.

% Archaeological Heritage, Digital Preservation, Virtual Environment, Photogrammetry, 3D Modeling, Responsive Website.

%  Digital Archaeology, Virtual Preservation, Cultural Heritage, Virtual Environment, Photogrammetry.