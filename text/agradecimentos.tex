Gosto de dizer que ninguém chega a lugar nenhum sozinho, logo, devemos reconhecer as pessoas que fizeram parte da nossa jornada e honrá-las! Hoje, ao olhar para trás, percebo que cada passo dado nesta trajetória só foi possível porque tive pessoas ao meu lado.

Primeiramente, quero expressar minha mais profunda gratidão à minha família, que não apenas me apoiou, mas foi o pilar que sustentou meus sonhos e me guiou até este momento. Sem ela, nada disso seria possível. Minha família é meu porto seguro, minha motivação diária. Seu apoio incondicional me fez seguir em frente. Agradeço especialmente à minha mãe. Sua sabedoria e dedicação são exemplos que carrego comigo todos os dias.

Ao meu orientador, professor Me. Afrânio Furtado de Oliveira Neto, sou eternamente grato pelo conhecimento compartilhado, pela orientação e pelas palavras de encorajamento nos momentos de incerteza. Seu bom humor e companheirismo tornaram esta jornada muito mais leve. Também agradeço ao meu co-orientador, Edson Borges, por abrir as portas para que eu trabalhasse com esse assunto fascinante que é o das artes rupestres e fotogrametria. Ele me mostrou a importância de preservar nossa história e me inspirou a trabalhar em algo tão relevante para a sociedade.

Minha gratidão se estende aos amigos que estiveram ao meu lado, torcendo pelo meu sucesso e me incentivando a seguir em frente. Ao Ramon, meu parceiro no PIBIC, agradeço pela parceria, pelas fotos, pesquisas e pelos momentos de descoberta juntos no sítio da Toca da Onça. Foi uma experiência inesquecível, e sei que formamos uma dupla que vai além do projeto.

Sou grato ao Naoki Miura, que além de meu irmão, é também meu colega de turma. E à minha amada Sara Cândido Fernandes, dedico palavras de carinho e gratidão. Vocês estiveram passando por todo esse processo junto comigo, e que mesmo com cada um ocupado com seu próprio TCC, se colocaram à disposição para me ajudar com \textit{feedbacks} e com a análise heurística.

Ao Instituto Federal de Goiás (IFG), minha segunda casa, sou imensamente grato. Ali, encontrei não apenas professores e colegas, mas amigos e uma verdadeira família. Cada corredor, sala de aula e laboratório guarda memórias que levarei para sempre comigo. Aos meus professores que me capacitaram profissionalmente, afiaram meu pensamento crítico e ampliaram meus horizontes, sou profundamente grato!

Não posso deixar de mencionar aqueles que me acolheram em suas casas quando precisei superar limitações técnicas. Ao Pedro Henrique Barros e ao Otávio Profeta, meu sincero obrigado por permitirem que eu utilizasse seus computadores para rodar programas de fotogrametria e Unreal Engine. Otávio me deixou até virar a noite na casa dele pra processar um dos modelos 3D. Vocês foram fundamentais para que eu pudesse avançar no projeto. E ao professor Leomar Rufino Alves Júnior, especialista em fotogrametria, minha admiração e reconhecimento por ter captado imagens com drones e me ensinado sobre renderização 3D.

Também quero agradecer aos proprietários da Fazenda Pedra, Sr. Luiz Fernando Lêdo e Dr. Mauro Passos, pela generosidade em disponibilizar o espaço para a realização deste trabalho.  Ao Guia e Condutor de Turismo de Aventura e Espeleoturismo, Noel José dos Santos, meu respeito e gratidão por nos conduzir em nossas visitas técnicas e nos mostrar a riqueza cultural e natural da região.

Por fim, agradeço a mim mesmo. Por não ter desistido. Por ter enfrentado os desafios com determinação e resiliência. Por acreditar que, mesmo nos momentos mais difíceis, valia a pena continuar lutando. Exercitar a gratidão ativa é celebrar a jornada tanto quanto o destino. Nos ajuda a sermos mais contentes com a vida e a superar as adversidades. 

Portanto, do fundo do meu coração, agradeço a todos que, direta ou indiretamente, contribuíram para que este trabalho se tornasse realidade. Que possamos seguir juntos a um futuro iluminado! 
\\
\\


\begin{flushright}
{\japanesefont ありがとうございます} \\
\textit{(Arigatou gozaimasu!)} \\
(Muito obrigado!)
\end{flushright}
