


Agradeço a minha mãe por me apoiar sempre, e por fazer eu chegar ate aqui. Sem ela não seria possível.
Ao meu orientador professor Dr. Waldeyr Mendes Cordeiro da Silva, pelo conhecimento, orientação e compreensão.
Aos meus amigos pelo apoio, incentivo e torcida pelo meu sucesso.
A todos que me ajudaram direta ou indiretamente nessa nessa jornada. 


Ninguém chega à lugar nenhum sozinho.
É importante reconhecer as pessoas que fizeram parte do processo
exercitar a gratidão ativa. "Citação de autor e pesquisa científica sobre gratidão"

Por isso, do fundo do meu coração gostaria de agradecer a cada um que ajudou direta ou indiretamente para que esse projeto fosse possível. 

Otávio Profeta por ensinar Inno Setup e emprestar os computadores para que eu pudesse rodar 
Ao Pedro e ao Otávio profeta, por me receberem em suas casas para rodar os programas de fotogrametria e Unreal quando hardware tive limitação de hardware.

Ao professor Leomar, por fazer a captura dos das fotos com drones e ensinar sobre a técnica de fotogrametria e renderização 3D.

A minha família pelo apoio, incentivo e torcida pelo meu sucesso.

Ao meu irmão e colega de curso Naoki Miura, por ser um parceiro e amigo.
A minha amada Sara Cândido Fernandes, pelo apoio e incentivo.

Ao meu orientador, sempre bem humorado, Afrânio Furtado, pelo companheirismo, apoio,  orientações e desorientações.

Ao meu co-orientador Edson Borges, por me mostrar as maravilhas da arte rupestre e ter proporcionado a oportunidade de eu trabalhar em algo tão interessante e relevante para a sociedade.

A minha segunda casa, também conhecia como Instituto Federal de Goiás, do qual tenho imenso carinho e gratidão por ter feito parte da minha história.

Aos meus professores que me forneceram a base de conhecimento nessa jornada. (Falar do marcos de alguma forma aqui).

